% =====
% Model and dynamics
% =====
\subsection*{Model and dynamics}

We study a family of regular polygons of side number $n=3-8$, shown in Fig. \ref{fig:model}A.
Polygons are modeled as hard particles that interact via a purely repulsive contact potential.
This contact is modeled with a Weeks-Chandler-Anderson (WCA) potential of radius $r=1$ \cite{WCA_1971}.
We set particle side length $a$ to maintain a constant $\frac{\sum_is_i}{2\pi r}=9$ for $r=1$.

We also explore anisotropy in the direction a constant active force ($\vec{F}_A$) applied to each particle.
We set $\vec{F}_A$ to be either perpendicular to a side of the particle (\textit{edge-forward}) or directed directly out a corner (\textit{vertex-forward}), with the orientation of the director locked to the particle.
Rotational noise of the particles allows the active force to diffuse with particle rotation. % implemented through the integrator

All simulations are run with $\Pe=150$, where $\Pe$ is a measure of active (advective) to diffusive motion. \textcolor{red}{Add formula for \Pe here.}
In this regime, thermal fluctuations contribute much less to the motion of the particles than does the active driving force.
To keep the $\Pe$ value constant across all simulations, the magnitude of the active force ($|\vec{F}_A|$) set to 1 and the temperature of the thermal bath governing the fluctuations is set as $T = \frac{|\vec{F}_A|*\sigma}{\Pe}$.

Particle motion is solved for using Langevin equations of motion.
The magnitude of rotational and translational noise is implemented in the integrator as a function of $T$.
The magnitude of the random force ($\vec{F}_R$) is chosen via the fluctuation-dissipation theorem to be consistent with the specified drag and temperature, $T$.
To closely approximate Brownian motion, mass of the particles is set to $m=1e-2$.

The Langevin equations of motion with an active driving force are given by:
$$ m\frac{\d\vec{v}}{\d t} = \vec{F}_C - \gamma\cdot\vec{v} + \vec{F}_R + \vec{F}_A $$
$$ \langle |\vec{F}_R|^2\rangle = 2dkT\gamma/\delta t$$

Here, $\vec{F}_C$ is the force on the particle from all potentials and constraint forces, $\gamma$ is the drag coefficient, $\vec{v}$ is the particle's velocity, and $d$ is the dimensionality of the system (2). 
We approximate the rotational friction drag coefficient in 2D, $\gamma_r$, as a function of the translational friction coefficient $\gamma$ per the Stokes-Einstein relationship: $\gamma_r = \frac{\sigma^2\gamma}{3}$.
We set the translational drag coefficient $\gamma$ to 1 and calculate $\sigma$ as the diameter of an equi-area disk to a given n-gon.


% =====
% Simulation details
% =====
\subsection*{Simulation details}

Each simulation is composed of $N=1e4$ particles, with 10 replicates run at each statepoint.
Particle positions are randomly initialized and allowed to relax with a spring potential at $\phi=0.10$ \cite{DPD_2011}.
The spring potential is then turned off and the WCA pair potential turned on (i.e. shape excluded volume) while the box compresses to the target system density.
Only after these initialization steps is the active force is turned on.
The direction of the active force is initialized randomly from the set of orientations of interest (either \textit{edge-} or \textit{vertex-facing}).

The simulation is run for $5000\tau$ with a timestep of $dt=1e-3$ (where $\tau = \frac{\sigma\gamma}{|\vec{F}_A|}$, the time it takes a particle to actively travel its own particle length).
We validate that simulations have reached steady state by the end of these simulations by validating that the system pressure has reached steady state (see SI Fig \ref{fig:SSpressures}).
\textcolor{red}{Check against this paper in case there are any claims I should be aware of: \cite{SolonEA_2015_NaturePhysics}.}
The system is initialized for $5e5$ time steps, and compressed to the target density for $5e5$ time steps.

Langevin dynamics simulations are performed on graphic processing units (GPUs) with the HOOMD-blue software package \cite{HOOMD_2008, HOOMD_2015}.
Shape interactions are modeled using the discrete element method implemented in HOOMD-blue \cite{DEM_2017}.

%TBD: How we determined the drag coefficients, rotational diffusivity to use \\
%\cite{BetEA_2016_arxiv}: Efficient shapes for microswimmers \\
%\cite{Purcell_1977_AmericanJPhys}: Life at low reynolds numbers \\


% =====
% Determining critical density
% =====
\subsection*{Determining critical density}

% Add critical density to methods, and why others from the literature can't work

There is not yet an accepted best-practice method for determining whether a system of active particles has clustered.
In one study, a system was considered ``clustered'' if the fraction of system particles in the largest cluster was ${\geq}0.2$.
However, this method is ill-suited for our systems, which include clearly phase-separated systems of many small clusters.
Next, other studies have used a local-density histogram of randomly-sampled points of the simulation box \cite{Bruss_2017_arxiv}.
If the histogram displayed two peaks, the system was considered phase-separated.
However, the very low system densities studied here limit the efficacy of this approach (i.e. at a packing fraction of 0.01, the high-density ``peak'' would be ${\leq}2\%$ the magnitude of the larger peak).

To determine phase separation even at low densities, we calculate two separate histograms of local densities at a radius of \textcolor{red}{X} from : (1) randomly sampled points ($N=1e5$) and (2) about each particle ($N=1e4$).
We then calculated a position-normalized local density histogram of the system by multiplying the frequencies of local densities in each bin by one another.
If the resulting histogram has a high-density peak ${\geq}20\%$ the height of the low-density peak, we consider the system to be phase-separated.

The fraction of systems that has phase-separated for a given system packing fraction is the fraction of the ten replicates run at each packing fraction which has phase separated, based on the above criteria.





