\documentclass[11pt, oneside]{article}   	% use "amsart" instead of "article" for AMSLaTeX format
\usepackage[margin=1in]{geometry}
\geometry{letterpaper}
\usepackage[parfill]{parskip}    % Begins paragraphs with an empty line rather than an indent
\usepackage{graphicx}				% Use pdf, png, jpg, or eps§ with pdflatex; use eps in DVI mode
\usepackage{hyperref}			% Allows me to link in the text with \href{link}{description}
\usepackage{amssymb}
\usepackage{amsmath}
\usepackage{mathtools}
\usepackage{fancyhdr}
\usepackage{xcolor}
\usepackage{titling}			% Allows me to set the \droptitle to make it less obnoxious
\usepackage{titlesec}			% Allows me to use \titlespacing command
\usepackage{blindtext}			% Allows me to generate Lorem Ipsum
\usepackage{pdfpages}			% Allows me to include pdf pages with the \includepdf[pages=-]{Filename.pdf} command

% Set text to 6 lines of 11pt font per 1 inch. Normal baseline skip is 13.6
% Source: https://tex.stackexchange.com/questions/23824/6-lines-in-one-inch
% 1 inch = 72.27pt
% First line is 11pts
% Remainder spread as (61.27/5)/13.6 = 0.901
\linespread{0.901}

% Spacing changes to make the document more compressed
\usepackage{enumitem}			% Allows me to set spacing to be more compressed for lists globally
\titlespacing{\section}{0em}{2em}{0em}		% Sauce: http://www.ctex.org/documents/packages/layout/titlesec.pdf, bottom p4
\setlist{nosep}								% Compresses all lists using enumitem package

% Headers
\pagestyle{fancy}
\fancyhf{}
\lhead{Shannon Moran}
%\chead{}
\rhead{Last updated: \today}
\rfoot{\thepage}

\title{Identifying and designing an information-driven approach for targeted colloidal self-assembly}
\author{Shannon Moran}
\date{February, 2018}






\begin{document}

% =====
% TITLE PAGE
% =====
\maketitle
\thispagestyle{empty}
\noindent
This paper is submitted in partial fulfillment of the University of Michigan Chemical Engineering Department Preliminary Exam requirements.
\\ \\
\textbf{Committee:} 
\begin{quote} 
Prof. Sharon Glotzer (Advisor) \\ Prof. Ronald Larson \\ Prof. Robert Ziff \\ Prof. Xiaoming Mao (Cognate: Physics)
\end{quote}
\par
Note: Total length must be less than 15 pages of text. Includes figures, excludes title page, list of references, and CV.


% =====
% CONTENTS
% =====
\newpage
\thispagestyle{empty}
\tableofcontents
\newpage
\listoffigures


% =====
% ONE PAGE SUMMARY
% =====
\newpage
\thispagestyle{empty}
\section*{Project Summary: ``Designing an information-driven pathway approach for targeted self-assembly''}

% Include a self-contained description of the activity proposed;
% potential hazards and safety precautions should be identified;
% max one page

Self-assembly is a powerful tool for creating complex materials with tailored particle interactions.
Systems order with interactions as simple as hard-particle excluded volume and as complex as DNA-programmed origami. 
While much research has been focused on understanding how to design highly-specific building blocks and predict assembled structure, relatively little work has been focused on optimizing specificity in self-assembly system design.

Put another way-- 
\textbf{what is the minimal set of instructions needed to achieve targeted, self-assembled complexity?}

To investigate this question, we use a system of folding nets (of the five Platonic solids) with non-specific edge interactions.
Previous molecular dynamics studies of this system have shown that compact nets with more leaves are more likely to be able to fold into (i.e. assemble) the Platonic solid from which they were derived (i.e. their target structure).   
Authors observe that in these nets, folding pathways are enabled by the formation of local, native bonds, mirroring phenomena observed in protein folding.

This suggests that though the interactions are not specific, the combination of geometry and attraction serve to make some of these bonds more effective than others.
In turn, this suggests that it is possible to identify bonds that are more critical to assembly than others. 
If we are able to rank the importance of interactions on the yield of an assembly pathway, then it stands to reason that we can determine which interactions are the minimally sufficient set needed to guide assembly into a target structure. 

\textbf{Project 1: Define a measure of pathway information}\\
Studies have successfully developed measurements of the specificity (and resulting information capacity) for specific interactions of lock and key pairs.
However, no such metric exists linking a bond to the self-assembly yield of a starting configuration.
We propose developing a metric that measures the likelihood of a bond being a part of a successful assembly pathway-- i.e. the amount of mutual information between the bond and the final assembled structure.

Using this method, we can identify the most critical bonds for assembly and develop heuristics for identifying them in un-studied nets.
We can test our hypotheses that certain bonds are the most critical for successful assembly by incrementally adding bonds identified as ``critical'' to nets known to have poor yield of their target structure.
In this way, we can then also develop a measure of information efficiency of a structure relative to its target structure.
How much information must we give a system (in the form of specified bonds) for a starting material to reach its target structure at a given yield?

\textbf{Project 2: Develop energy landscapes for identifying kinetic barriers to assembly} \\
We hypothesize that seeding a structure with critical bonds can help a net avoid searching local minima en route to the global minimum. 
We can identify these local minima by building net disconnectivity graphs which convert a net's folding energy landscape into an easily-digestible network of free-energy minima and transition states.
Using these disconnectivity graphs, we will look for features of good- and poor-folding nets. 
Additionally, studying optimal energetic pathways may give us further insight into why nets with more leaves fold better than others.

\textbf{Project 3: Design pluripotent nets from minimal instructions} \\
Finally, understanding the minimum amount of instruction needed to drive a configuration to a given assembly opens the possibility of embedding instructions for multiple states into a starting material.
We call such a material pluripotent. 
Given the rules for minimal assembly instructions from Project 1 and the energy landscapes developed in Project 2, we will attempt to use a minimum set of specific bonds to embed multiple potential target structures into a net.



% =====
% MAIN TEXT
% =====
\clearpage
\setcounter{page}{1}


% =====
% MOTIVATION AND LITERATURE REVIEW
% =====
\section{Introduction and motivation - 1 pg}

When we think about the major challenges facing materials science, we are fundamentally faced with this idea of inversely designing materials.
That is, I decide that I want to create a material that behaves your sweat-wicking shirt under one condition, stiffens under another, and when given a particular stimulus can reconfigure its structure.
Currently, if I wanted to make a material like that for you, I'd naively take materials that have each of those properties and figure out how I could get them to work together.
Or, I'd look for novel materials that have properties close to those of the material I want to make.
We would call this designing the material.

This is inefficient.
In the inverse design problem, we take the properties we want and create the materials that will give us those properties.
Machine learning and materials science are coming together on the active front of this research.
However, being able to predict, even perfectly, what can be made from existing materials by definition limits us to the set of materials that currently exist.
This is a known challenge in materials science-- how do we probabilistically explore phase space outside of phase space where we have data?
While intriguing in its own right, that is not the topic of the thesis proposed here.

Instead, we might think about this from a fundamental physics point of view.
If we want to make complex materials that have embedded stimuli responses, or assemble into a specific target structure, we must give the building blocks of such complex materials some amount of direction.
We can think about this amount of direction as an amount of \textit{information}.

This is not to say that we are looking to have building blocks act as storage devices, as in \cite{Phillips_2014_SoftMatter}.
In that work, each building block is a cluster of multiple particles in whose arrangement can be stored a ``high density'' of information.

Similarly, in a recent proposal between our group and those of Marke Bathe (MIT), Mawgwi Bawendi (MIT), and Oleg Gang (Columbia), we proposed a biosynthetic, high-density storage structure composed of DNA nanocubes (Figure \ref{fig:semisynbio}).
Within these nanocubes, information could be stored in the different dies intercolated into the frame of the cube, into quantum dots placed into the frames, or even in the shape of the frames themselves.
If we them move a level higher, we can imagine storing additional information in the order of these nanocubes relative to one another.

However, using these and solutions like them for high-density storage requires us being able to write, read, and store information into these formats.
Fundamentally, these three challenges are predicated upon the ability to specifically place blocks where they need to go (``write'').
Current methods include sonic and laser tweezers (manual), specific DNA interactions (energetic), or incremental addition (kinetic).
How do we compare between these methods, though?

Here, I propose that the ability of building blocks to form a target structure can be distilled down into a concept of \textit{information}.

This is not a new concept.
In our group, we are comfortable with the concept that a target structure is the result of a minimization of free energy.
In systems devoid of inter-particle forces, this then reduces down to a maximization of entropy.

Statistical mechanical ``entropy'' shares its name with information ``entropy'' in communications theory.
While this directly came about because of the form similarity between the two, much energy since has been devoted to developing frameworks connecting the two.
Jaynes, in the 1950s, spent two long articles trying to reconcile the two. 
Books, and multiple articles, have been dedicated to explaining why these concepts are similar.

While much time has been spent developing the theory, very little time has been spent directly leveraging this concept for embedding information in systems governed by statistical mechanical ensembles-- such as colloidal-scale self-assembly.

Key line from Simons proposal: ``A coherent framework of thermodynamic and non-equilibrium processes seen through information theoretic eyes could lead to new theories for encoding information in matter-- which would allow for the design of novel materials and novel material behavioral control.''


New outline: 
\begin{itemize}
\item Self-assembly in materials science can lead to a variety of structures, complexity
\item The future of materials science relies upon inverse design
\item Inverse design requires understanding how the building blocks of a material inform its overall structure
\item Thinking more simply about tailored self-assembly... we want to direct the behavior of a system
\item We can think of this as giving the system some amount of information: binding preferences, kinetics, etc
\item Lots of work has gone into trying to understand how to measure and then most efficiently provide systems with that direction (will detail in the background section)
\item You know what does this really well, though? Proteins in nature
\item Proteins conformationally change while binding; lots of complexity
\item Here we use a simpler system of folding nets, which actually has much less complexity
\item The overarching challenge is: what is the most efficient way of assembling a given structure?
\item Such a question could motivate a career, so we will further limit this scope in the coming pages.
\end{itemize}

% 1 page statement of the problem, purpose and significance of the research

\newpage
%\section{Literature Background - 2-3pg}

\subsection{What is information?}
% Good quora: https://math.stackexchange.com/questions/331103/intuitive-explanation-of-entropy

Let's first look at information in the context of communications theory.

The \textit{information}, $I$, we get from an event happening is given by:
\begin{equation}
I(p) = -\log{(p)}_b
\end{equation}

where $p$ is the probability of an event happening and $b$ is the base.
Base 2 is commonly used in information theory, and forms the unit of information.
For instance, the unit of base 2 information is a bit, base 3 are trits, base 10 are Harleys, and base $e$ are nats.

In 195X, Claude Shannon also extended this concept by introducing the concept of \textit{information entropy}.
In this context, entropy is the average (expected) amount of information gained from a given event.
Specifically, for an event with $n$ different outcomes this can be written as:
\begin{equation}
\text{Entropy} = \sum_{i=1}^{n}p_i\log{p_i}
\end{equation} 

For a discrete random variable $X$ with $p(x)$, the entropy can be written as:
\begin{equation}
H(X) = \sum_x p(x)\log{p(x)}
\end{equation}


Entropy does not range from 0 to 1.
The range is set based on the number of possible outcomes $n$, i.e. $-\leq\text{Entropy}\leq\log{(n)}$.
Entropy is equal to 0 (minimum entropy) when one of the probabilities is 1 and the rest are 0's.
Entropy is $\log{(n)}$ (maximum entropy) when all the probailities have equal values of 1/n.

In the case of designing specific outcomes for an event, then, we want to minimize the entropy along each leg of the pathway leading to an event.
Put another way, we want to maximize the probability that the event will proceed down the pathway we want it to.

However, the concept of ``information'' in this context is then counter-intuitive.
Information in communication refers to how likely an event is.
When a rare event happens, we gain more ``information'' from that event.
However, in the context of designing specific outcomes, we are not looking to read out bits of information once an event has happened.
We are looking to design the likelihood of an event occurring.

% Might want to include a ball-selection example here: http://www.csun.edu/~twang/595DM/Slides/Information%20&%20Entropy.pdf

In the words of MIT professor C\`{e}sar Hidalgo, ``It is hard for us humans to separate information from meaning because we cannot help interpreting messages.'' 
We face the same problem here-- by saying that a pathway has more information than another, we are implicitly saying that it is a rarer event than a lower-information pathway.

Counter-intuitively, in designing pathways for self-assembly, then, we are looking to design minimum-information pathways.
\textbf{However, in aligning with our intuition from self-assembly, this means we are looking to maximize the entropy of an assembly pathway.}

However, we can use the concept of \textit{mutual information} in defining how much information is stored in an interaction in an intuitive manner.
(See notes on the Brenner paper below.)
Mutual information $I(X;Y)$ is a global measure of interaction specificity in systems with many distinct species.
It quantifies how predictive the identity of a lock $x_i$ is to the identity of key $y_i$ found bound to it.

\subsubsection{How does this tie into statistical mechanics?}


\subsection{What is information, in the context of self-assembly?}

Let's first look at a paper from the Brenner group, the ``Information capacity of specific interactions' \cite{Huntley_2016_PNAS}.
Their main thesis is that specific binding interactions have energetics that allow binding to occur with measurable probability.
Thus, we can measure the relative information in different types of binding.
This is more in line with the communication theory view of information (rare events giving more information) than it is with the materials view of information, in which high information events imply high probability of a desired event happening.

Our group, and many others in the materials community, are looking to engineering materials to control their structures, behaviors, etc.
A common method of engineering these materials is by tailoring the interactions between their components through chemistry, shape, etc.
By understanding how much \textit{assembly information} can be contained in these interactions, we can:
\begin{enumerate}
\item Compare the efficacy of different types of interactions in delivering desired behavior(s)
\item Theoretically predict the efficacy of new types of interactions
\end{enumerate}

Let's take the example of a lock and key system.
\textcolor{red}{ADD SECTION ON BRENNER PAPER FROM LIT REVIEW LAST YEAR}


Any of Jacobs' papers that talk about this?: uses connectivity graphs


\subsection{Where have folks applied directed self-assembly? Why do we care about it?}
% i.e. what is the real-world problem we're actually trying to solve?

Glotzer, Kotov - self-assembly

Mirkin - experimental, DNA-directed

Kamien - kirigami

Glotzer, Desmaine - folding

Frenkel, Jacobs - pathway design

Wales - pathway designs, disconnectivity graphs


\subsection{Already proposed work: Semiconductor Synthetic Biology}

Include work done on the NSF grant proposal: Using DNA-mediated assembly to store information in nanoparticle arrays.

Specifically, this is an example of 1b): addressable complexity, then trying to engineer how to get the particles to where they should go in the most energetic and information/complexity-efficient manner possible.







% Literature survey and description of research already performed by the applicant

% =====
% RESEARCH PROPOSAL
% =====
\section{Description of proposed research (7-8 pg, 2-3 pg per aim)}
% Including method or approach and expected difficulties
% This must constitute about 50\% of the text of the written proposal: 7-8 pages
% Clear statement of the work to be undertaken and must include:
% Objectives for the period of the proposed work and expected significance
% Relation to the present state of knowledge in the field and to work in progress at Michigan/elsewhere
% Expected research program sequence
% Decision points expected during the course of the research
% Methods of data reduction, evaluation, interpretation and presentation

``Information'' are those factors that impact the yield and kinetics of self-assembly (thermodynamics of the free energy landscape and the kinetics of the path to get to a target structure from a given starting point).

Specifically, as we look to both understand how nature governs self-assembly into target structures, we need a language to understand this.`
Nature is very good at already picking an optimized route through a free energy landscape \cite{Jacobs_2016_BiophysicalJournal}.

\subsection{What I want to accomplish in my thesis}

1) Can we figure out which bonds/interactions are the most critical in an assembly process or an assembly pathway?
We intuitively know that in 
Need transition state or pathway sampling methods (or might actually be able to get this just from Paul?s data? That would be sick)

2) With this metric, can we then define and minimize an information efficiency, e.g. the amount of bond specificity we need across the system to get a given success rate of assembly?
E.g. is it more efficient 
3) Given all this, can we then use machine learning to determine a priori the ?most efficient? level of specificity for self-assembling a target structure?
How to do pathway design is kind of an open question
We can find feature correlations, like Paul did
There?s also an approach called ?Computable Information Density? published by some colleagues (Chaikin) on the arxiv last August
Basic idea is that you can (1) somehow represent your system as an array of information which you can (2) run though a compression algorithm and (3) the ?information? is just the length of that compressed information
Would be really interesting to see if I could extend that idea to the features of an assembly system? in this case, nets? and 


1) Define a measure of pathway information. \\
\begin{itemize}
\item We already have ways of measuring how good a particular bond is
\item Are there particular bonds/connections that are the most important to get correct to enable forming the desired final structure?
\item  
\end{itemize}



2) Use that measure of pathway information to design ideal pre-cursors for target structures. \\



3) Attempt to use machine learning to predict ideal pre-cursors for given target structures. \\
\begin{itemize}
\item \cite{Long_2014_JPhysChemB}: Nonlinear Machine Learning of Patchy Colloid Self-Assembly Pathways and Mechanisms out of the Furguson group
\end{itemize}


4) Why is it important we find the ``most important'' pathway points? from \cite{Stern_2017_arxiv} \\
How then can self-folding origami be folded with a
minimal number of actuators? A lesson can be drawn
from similar glassy landscape search problems in models
of protein folding (e.g., Levinthal?s paradox [17, 19, 20,
41]) and related NP-hard satisfiability (SAT) problems
[21, 42] that vary from the Traveling Salesman Problem
to Sudoku [43]. A common element in these satisfiability
problems is that random seeding of the search for
the global minimum leads to repeated backtracking after
reaching local minima, both in the context of computer
algorithms (as the DPLL algorithm for k-SAT [21]) or for
physical dynamics (as in protein folding) [42]. However,
careful seeding of the search - e.g., if the right boxes are
filled in first in Sudoku [43] or if the right parts of the protein
are folded first - can greatly reduce or even eliminate
backtracking [21] before reaching the global minimum.
Correct seeding is even more critical for origami since
folding is assumed to happen at ?zero temperature? (e.g.,
without any noise or fluctuations). As a result, the structure
cannot backtrack out of a local minimum as in the
case of non-zero temperature SAT problems [42].

This reference also has a really good introduction section relating origami and self-assembly \cite{Stern_2017_arxiv}.


% =====
% PRIOR WORK: SEMISYNBIO PROPOSAL
% =====
\subsection{Collaboration: Synthetic biology memory}

The below was a response to an NSF call for proposals for a Semiconductor Synthetic Biology.

We proposed using DNA-mediated assembly to store information in nanoparticle arrays.

Specifically, this is an example of addressable complexity, then trying to engineer how to get the particles to where they should go in the most energetic and information/complexity-efficient manner possible.


\textit{From intro}:
In Aim 1, we will investigate monomeric block formation, exploring the self-assembly of arbitrary geometric DNA objects with incorporated optical elements that can be manufactured as information carriers, while allowing for superstructure formation through DNA-sequence barcoding. We will explore static assembly of 1D arrays of such DNA nanoparticles integrated with Memory Blocks (DNAMB) for encoding bitstream information that can be read out by fluorescence and electron microscopy. In Aim 2, we will explore 2D and 3D assembly, investigating techniques to algorithmically assemble and read out digital 2D and 3D information using optical and tomographic methods. In Aim 3, we will use molecular decision computing to assemble distinct, alternative lattices based on specific external signals. These results will offer the ability to encode and decode arbitrary datasets in ultra-dense molecular hard-drives, with environmental sensing and recording.

Aim 2. Dense, programmable molecular memory in 2D and 3D bit module lattice assemblies
Overview \& Rationale. Nanoparticle self-assembly depends on a balance of interaction forces, entropic effects, and system kinetics37,53,69-73. We can leverage these properties to direct self-assembly of shaped DNA nanoparticle into 2D and 3D arrays by controlling the position and valency overhangs that provide connectivity between DNA nanoparticles. Wireframe structure of DNA particle is highly suitable for encapsulation of memory blocks (e.g. Au NP, QDs, fluorescent dyes) and creation of DNAMB, a pixel in 2D or 3D arrays. To achieve information storage capabilities, it is required to investigate how the connectivity properties of DNAMB can be translated into their designed arrangement in the information- storing arrays. To self-assemble these systems into 2D and 3D ultra-dense data blocks, we will investigate the minimum interaction specificity needed to direct self-assembly into high fidelity ordered 2D and 3D arrays. We will also explore information retrieval from these arrays in 2D and 3D within pixels consisting of 1x1, 2x2, 4x4, etc., nanoparticle block arrays. In addition, we will establish methods for generating robust memory arrays that can preserve information under extreme conditions.

\textit{Proposed research, Aim 1}: Computationally, Glotzer and colleagues will develop a bit module interaction model to study the role of the DNA linkages on DNA cage self-assembly. Specifically, previous work on modeling solid particles with DNA-facilitated attraction37 will be extended to model the DNA cages that will be experimentally made by Gang, and it will consider realistic features of nanoparticle systems82,83. With this model in place, we can then extend the framework of digital alchemy, which treats particle properties as a thermodynamic variable, to particle interactions (here, DNA linkages)54. In this way, we can inversely design ideal DNA cages (e.g. shape, patchy interactions) that will robustly assemble a target structure. We will seek to balance site specificity without being overly unique?that is, design the highest information interactions that will allow for the minimum amount of linkage specificity for directing self-assembly84,85. This computational framework for the inverse design of bit packages that will assemble a given structure will enable high- throughput screening of particles of interest and serve as the basis for complex hierarchical structure and array assembly in the remainder of Aims 2 and 3.

Sub-Aim 2.2. Hierarchical assembly logic for higher-dimensional information storage
Overview. In Sub-aim 2.1, we explored approaches to assembling DNA frames into target 2D and 3D assemblies. Next, we precisely order ?bit modules??that is, DNA cages carrying functional particles? into arrays of discrete information. Toward this end, we design modules that carry the minimal information needed to reach target arrangements through a combination of particle anisotropy and DNA linkers. DNA
 computing groups have previously used DNA linkers to self-assemble complex 2D patterns6 and 3D shapes9. Here we extending these approaches to realize hierarchical 3D nanoparticle assembly design so that pixelated images act as dense data storage units. We will explore several complementary approaches to hierarchical assembly engineering, including sequential nanoparticle addition and ?one-pot assembly?, each of which will be explored together with inverse computational design of self-assembly pathways and particle geometries to achieve a robust assembly of designed arrays. Using the optical characterization strategies from Sub-aim 2.1, we will decode the information encoded in the structure, and probe sources of error and information loss in the self-assembly and read-out processes.

\textit{Proposed research}. Assembly of encoded 3D arrays can be approached in two strategies, or a combination thereof (Fig. \ref{fig:semisynbio}). In an entirely ?one-pot? assembly of a 3D array, all modules are linked with a large binding sequence set that has been fully computationally defined. Such an approach requires an enormous number of unique binding sequences, and even if fully defined, can run into high error rates when considering the assembly and packing of large (as compared with molecular assembly) and charged modules and/or materials. A second approach using sequential binding based on module groups of similar binding layouts requires less sequence diversity and can be automated using robotic liquid handling. However, this is vastly more process- and time-intensive than one-pot assembly. This approach represents hierarchical assembly, whereby 1D structures (?strings?) would be formed from the modules, 2D planes formed from the 1D libraries, and finally 3D encoded arrays from stacking of selected planes. An optimal assembly process that balances fully-encoded organization with direct addition of binding components would offer a hybrid approach of hierarchical assembly with sequential addition of groupings of computationally defined structures. Each of these strategies will be explored in this aim, using a combination of high-throughput, structure-based computational modeling and experiments.

The Glotzer group will extend their digital alchemy framework to probe diverse DNA linkage sequences and conjugation designs to realize specific, targeted inter-particle interactions. In addition, they will explore the roles of these interactions on the kinetics of array assembly to enable pathway design \cite{Jankowski_2012_SoftMatter} into desired arrays while avoiding undesirable ?side products?. In this way, we will explore computationally the interplay between the two extremes of one-pot and sequential assembly, and identify which combinations provide lowest assembly error while minimizing both assembly time and the number of required unique binding sequences. The Gang group will employ a home-built robotic system for automatic synthesis and assembly of DNAMB; that will allow establishing practical methods for creation of large number of diverse blocks required for the hierarchical assembly. While such approaches have been applied to molecular systems, they have not yet been realized for DNA frames integrating inorganic NP. To implement complementary pathway design strategies, Gang will fabricate DNA frames with thermally differentiated inter-vertex hybridizations to promote highly specific assembly path during thermally-driven self- assembly. For example, DNAMB strings will be assembled at higher temperatures, and planar and 3D arrays assembled at lower and lowest temperatures, respectively. We will use SAXS and tomography methods to reveal the pathway-controlled assembly process. The computational design of frames and pathways will be performed jointly between the Bathe, Glotzer, and Gang labs.

\begin{figure}[t]
\begin{center}
\includegraphics[width=6.5in]{../figures/SemiSynBio.pdf}
\caption{
(Lifted from SemiSynBio Proposal) Strategies for building information encoded 1D, 2D and 3D arrays. Sequential operations are very deterministic and can be carried out by automated robotic equipment, but even so are heavily process intensive and require many individual assembly steps. One-pot systems can be fully computationally defined, though in practice are heavily sequence intensive and be subjected to errors more readily than in molecular- scale systems when accounting for kinetic and thermodynamics of packing larger objects and materials. A hierarchical assembly methodology offers a hybrid approach of both strategies, where a sequential addition of structures preformed in a one-pot setup provide the desired 3D material organization.}
\label{fig:semisynbio}
\end{center}
\end{figure}

% =====
% PRIOR WORK: ACTIVE SHAPES
% =====
%\section{Prior work: Leveraging anisotropy for tailoring self-assembly in active systems}

The following is taken from an in-preparation manuscript. \cite{Moran_2018_unpublished}

\textbf{Relation to proposed thesis topic}: If we define information broadly as any quality of a building block that impacts the emergent behavior of a system of those particles (in this case, force direction and shape), then we can argue this work is looking at a few aspects of information in active systems.


\subsection{Background}

\subsection{Methods}

\subsection{Results}

\subsection{Next steps}


\begin{figure}[t]
\begin{center}
\includegraphics[width=5in]{../figures/Fig1.pdf}
\caption{\textbf{Model system}: (1) We use rounded shapes of constant S-C ratio, where the corners are rounded by a WCA potential, to ensure we can distinguish between shape steric (anisotropic) effects and isotropic behavior; (2) Force can be applied either perpendicular to the face or directed out a corner}
(Qs) should all particles be the same size?-- can't be, the way it's set up; all particles have drag of an equivalent disk; should we neglect noise? what does that mean for these simulations?
\label{fig:model}
\end{center}
\end{figure}

\begin{figure}[t]
\begin{center}
\includegraphics[width=6.5in]{../figures/Fig2.pdf}
\caption{\textbf{Critical density and nucleation behavior}: (A) Average domain size (cluster size? grain size?) versus time is different for disks versus shapes, and also depends on force director. (B) Critical density, the density at which SOME DEF OF CLUSTERING OCCURS, depends on both shape and direction of force director.}
\label{fig:phase_diagram}
\end{center}
\end{figure}

\begin{figure}[t]
\begin{center}
\includegraphics[width=6.5in]{../figures/Fig3.pdf}
\caption{\textbf{Collision efficiency}: Something with pressure? Not sure how to use this yet, but feel like there's something here...}
\label{fig:pressure}
\end{center}
\end{figure}


\begin{figure}[t]
\begin{center}
\includegraphics[width=6.5in]{../figures/Fig4.pdf}
\caption{\textbf{Displacement fields}: In contrast with disks, clusters are able to convert translational forces into rotation (highlighted in red boxes). Clusters of disks can?t sustain translational or rotational motion? clusters of shape can (this was off-hand noted in Suma et al)
}
\label{fig:velocity}
\end{center}
\end{figure}


\section{Time table}

\begin{figure}[t]
\begin{center}
\includegraphics[width=6.5in]{../figures/gantt.pdf}
\caption{Key milestones and tasks from Preliminary Exam through target defense date.}
\label{fig:gantt}
\end{center}
\end{figure}

See Figure \ref{fig:gantt} for key tasks and milestones through 2020, based on the projects outlined in the above sections.

\section{Conclusions and potential impact}

Ultimately, tailoring self-assembly and addressable complexity are driven by the question:

\begin{center}How do I give particles in my system sufficient information to find their proper position in a target whole?\end{center}

In systems targeting particular structures, tailored attraction (experimentally via single-stranded DNA or theoretically via an attractive potential) and repulsion can be used to tailor the structure of particle assembly.
In systems targeting addressable complexity with DNA origami and protein folding, such direction can be given with exact detail through genetic code.
Work by Brenner and Manoharan has sought to quantify the amount of specificity bonds in such systems can contain, and the limits on target system complexity given differing bond informations.

However, missing in this conversation is a measure of the importance of each individual interaction on the emergent behavior of the whole system.
I intend to contribute in part during the remainder of my PhD.
Instead of simply observing emergent behavior as an outcome of specifically-designed behavior of individuals, we could instead engineer such behavior as a quantifiable outcome of the interaction of an information-rich network of agents.
Having such knowledge may allow us to embed multiple ``states'' in a material, such as the minimum folds to move between two target origami structures \cite{An_2011_Robotica}.
Being able to measure, and ultimately control, the assembly directions stored in a starting material is a critical step to enabling materials by design and advancing materials science.

% ======
% BIBLIOGRAPHY
% ======
\newpage
\bibliographystyle{unsrt}
\bibliography{../../library/_PrelimBibliography}

\end{document}