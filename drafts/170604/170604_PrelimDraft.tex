\documentclass[11pt, oneside]{article}   	% use "amsart" instead of "article" for AMSLaTeX format
\usepackage[margin=1in]{geometry}
\geometry{letterpaper}
\usepackage[parfill]{parskip}    % Begins paragraphs with an empty line rather than an indent
\usepackage{graphicx}				% Use pdf, png, jpg, or eps§ with pdflatex; use eps in DVI mode
\usepackage{hyperref}			% Allows me to link in the text with \href{link}{description}
\usepackage{amssymb}
\usepackage{amsmath}
\usepackage{mathtools}
\usepackage{fancyhdr}
\usepackage{titling}			% Allows me to set the \droptitle to make it less obnoxious
\usepackage{titlesec}			% Allows me to use \titlespacing command
\usepackage{blindtext}			% Allows me to generate Lorem Ipsum
\usepackage{pdfpages}			% Allows me to include pdf pages with the \includepdf[pages=-]{Filename.pdf} command

% Set text to 6 lines of 11pt font per 1 inch. Normal baseline skip is 13.6
% 1 inch = 72.27pt
% First line is 11pts
% Remainder spread as (61.27/5)/13.6 = 0.901
\linespread{0.901}

% Spacing changes to make the document more compressed
\usepackage{enumitem}			% Allows me to set spacing to be more compressed for lists globally
\titlespacing{\section}{0em}{2em}{0em}		% Sauce: http://www.ctex.org/documents/packages/layout/titlesec.pdf, bottom p4
\setlist{nosep}								% Compresses all lists using enumitem package

% Headers
\pagestyle{fancy}
\fancyhf{}
\lhead{Shannon Moran}
\chead{Thesis Proposal DRAFT}
\rhead{Last update: \today}
\rfoot{\thepage}

\title{[Entropy transfer (thermodynamic and information) in materials systems]}
\author{Shannon Moran}
\date{}

\begin{document}

\maketitle
\thispagestyle{empty}
\noindent
\textbf{Committee chair:} Prof. Sharon Glotzer \\ \textbf{Committee members:} \\ Prof. Michael Solomon \\ Prof. Ronald Larson \\ Prof. Xiaoming Mao
\\ \\
\noindent 
This paper is submitted in partial fulfillment of the University of Michigan Chemical Engineering Department Doctoral Candidacy Exam requirements.
\\ \\
Total length must be less than 15 pages of text. Includes figures, excludes title page, list of references, and CV.
\newpage


Thesis: Information can be encoded and retrieved from colloidal materials to guide their behavior. \\
Working title: Working towards pluripotent materials \\
Why this \textit{new}? Why is this \textit{important}?

\section{Introduction and motivation - Non-equilibrium self-assembly}

Reconfigurable systems may be the key for unlocking adaptive material applications

Equilibrium self-assembly is useful for fabricating ordered structures on the nanoscale

However, equilibrium self-assembly processes aren't necessarily reliable 

Biology is really good at self-assembly, but it does so in a non-equilibrium manner in which systems are driven out of equilibrium by a constant input of energy (ATP)

However, the fundamental physics of such driven non-equilibrium self-assembly remains poorly understood

However, we have a problem-- we don't really understand non-equilibrium self-assembly, because we don't really understand non-equilibrium thermodynamics

KEY: 
Reconfigurability-induced switching of a material's structure could be explored as a way of storing information (``memory'') in a material

\section{Introduction and motivation - Information}
% Statement of the problem, purpose and significance of the research

Entropy is primarily an equilibrium construct. Information theory, however, is considerably broader. 

Open question: can these three distinct meanings of entropy-- information, number of accessible states, and heat-- be brought together in a meaningful way that would allow us to predictively and reliably encode information in materials systems?

1) What is entropy: Thermodynamics / heat - Entropy production out of equilibrium
- Energy is dissipated as entropy
- ``Entropy product is believed central to the behavior of non-equilibrium and dissipative systems, where structures are stabilized by the consumption of energy and the production of entropy
- General principles and predictive theories are not in hand because of our inability to mathematically formulate non-equilibrium variational principles and because we lack appropriate geometric and topological measures of entropy and information-- what would these even look like?

2) What is entropy: Statistical mechanics / number of accessible states - Equilibrium thermodynamics
- Boltzmann entropy; probability of being a given microstate is a function of the prevalence of that microstate

3) What is entropy: Information
- Story of Claude Shannon developing his information theory
- The fact that he named it entropy wasn't necessarily based on any explicit relationship to the thermodynamic or statistical mechanical form of entropy

4) In the granular materials community, ``memory'' is embedding some replicable response pattern in 

5) Engineering application: Designing functional, reconfigurable materials will require some method of storing information in a material-- we might call this ``memory''

6) Scientific application: Understanding how to do this is a fundamental problem
- Understanding the linkage between these three is critical to a deeper understanding of biochemical processes, out of equilibrium dynamics of macroscopic systems, and the ``dynamics'' part of thermodynamics

Key takeaway from information proposal: A coherent framework of thermodynamic and non-equilibrium processes seen through information theoretic eyes could lead to new theories for encoding information in matter-- which would allow for the design of novel materials and novel material behavioral control.

\section{Background - Literature review}
% Literature survey and description of research already performed by the applicant


\section{Background - Research already completed}
\subsection{Role of shape in collective motion}
Cluster onset

``Steric bonding''

\section{Description of proposed research (7-8 pages, 2-3 p per goal)}
% Including method or approach and expected difficulties
% This must constitute about 50\% of the text of the written proposal: 7-8 pages
% Clear statement of the work to be undertaken and must include:
% Objectives for the period of the proposed work and expected significance
% Relation to the present state of knowledge in the field and to work in progress at Michigan/elsewhere
% Expected research program sequence
% Decision points expected during the course of the research
% Methods of data reduction, evaluation, interpretation and presentation
\subsection{Aim 1}
For each aim:
\begin{itemize}
\item Goal and significance
\item Hypothesis
\item Approach, methods, analysis to be used (including relevant citations)
\end{itemize}

We must both define and ascertain the information content of a self-assembly pathway
If self-assembly proceeds deterministically along one route only, the entropy is clearly zero
However, optimal self-assembly processes are likely to have a high, maybe maximal ``path entropy''

To realize such information-rich structures in a wide range of chemical or colloidal structures, we must have a deeper, more rigorous understanding of the factors that optimize the yield and the kinetics of self-assembly

We could both solve the entropy problem and enable a completely new field based on the encoding of information in physical structures

\section{Time table}
% Basically a Gantt chart
\textit{Will put together a Gantt chart for each project with associated milestones. 	}

\section{Conclusion}
Instead of simply observing emergent behavior as an outcome of collective motion of individuals, we could instead engineer such behavior as a quantifiable outcome of the interaction of an information-rich network of agents.

\end{document}