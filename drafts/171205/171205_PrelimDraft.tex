\documentclass[11pt, oneside]{article}   	% use "amsart" instead of "article" for AMSLaTeX format
\usepackage[margin=1in]{geometry}
\geometry{letterpaper}
\usepackage[parfill]{parskip}    % Begins paragraphs with an empty line rather than an indent
\usepackage{graphicx}				% Use pdf, png, jpg, or eps§ with pdflatex; use eps in DVI mode
\usepackage{hyperref}			% Allows me to link in the text with \href{link}{description}
\usepackage{amssymb}
\usepackage{amsmath}
\usepackage{mathtools}
\usepackage{fancyhdr}
\usepackage{titling}			% Allows me to set the \droptitle to make it less obnoxious
\usepackage{titlesec}			% Allows me to use \titlespacing command
\usepackage{blindtext}			% Allows me to generate Lorem Ipsum
\usepackage{pdfpages}			% Allows me to include pdf pages with the \includepdf[pages=-]{Filename.pdf} command

% Set text to 6 lines of 11pt font per 1 inch. Normal baseline skip is 13.6
% 1 inch = 72.27pt
% First line is 11pts
% Remainder spread as (61.27/5)/13.6 = 0.901
\linespread{0.901}

% Spacing changes to make the document more compressed
\usepackage{enumitem}			% Allows me to set spacing to be more compressed for lists globally
\titlespacing{\section}{0em}{2em}{0em}		% Sauce: http://www.ctex.org/documents/packages/layout/titlesec.pdf, bottom p4
\setlist{nosep}								% Compresses all lists using enumitem package

% Headers
\pagestyle{fancy}
\fancyhf{}
\lhead{Shannon Moran}
\chead{Thesis Proposal DRAFT}
\rhead{Last update: \today}
\rfoot{\thepage}

\title{[Working towards pluripotent materials]}
\author{Shannon Moran}
\date{\today}

\begin{document}

\maketitle
\thispagestyle{empty}
\noindent
Broad thesis: Information can be encoded and retrieved from colloidal materials to guide their behavior.
%This paper is submitted in partial fulfillment of the University of Michigan Chemical Engineering Department Doctoral Candidacy Exam requirements.
\\ \\
\textbf{Committee chair:} Prof. Sharon Glotzer \\ \textbf{Committee members (currently):} \\ Prof. Michael Solomon \\ Prof. Ronald Larson \\ Prof. Xiaoming Mao
\\ \\
Note: Total length must be less than 15 pages of text. Includes figures, excludes title page, list of references, and CV.



\newpage

% Why is this \textit{new}? Why is this \textit{important}?

\section{Introduction and motivation (1pg)}
% Statement of the problem, purpose and significance of the research


Thesis: Reconfigurability-induced switching of a material's structure could be explored as a way of storing information (``memory'') in a material. We can consider this pluripotency.

Goal: Predictively and reliably encode information in materials systems. (I know this is too broad)

Problem statement:
\begin{itemize}
\item \textit{Engineering}: Designing functional, reconfigurable materials will require some method of storing information in a material-- we might call this ``memory''
\item \textit{Science}: Understanding how to ``store information'' in a material that embeds a response is a fundamental problem; further understanding the linkage between information-theoretic entropy and thermodynamic entropy concepts (accessible states, heat) could further our understand of dynamics of biological, etc processes
\end{itemize}


Ideas:
\begin{itemize}
\item Reconfigurable systems may be the key for unlocking adaptive material applications
\item At a particle level, we can think of this as building pluripotent building blocks that contain some response to a stimulus
\item At a system level, we can think of this as having metastable configurations in response to some stimulus
\item At an assembly-level, we can think of this as building blocks that can be engineered to form specifically-ordered arrays
\end{itemize}


Key line from Simons proposal: ``A coherent framework of thermodynamic and non-equilibrium processes seen through information theoretic eyes could lead to new theories for encoding information in matter-- which would allow for the design of novel materials and novel material behavioral control.''

\section{Background - Literature survey (2-3 pg)}
% Literature survey and description of research already performed by the applicant

Placeholder.

\section{Background - Research already completed (2-3 pg)}
\subsection{Role of particle shape on the emergent behavior of active systems}
Take selections from in-preparation paper.

\textbf{Relation to proposed thesis topic}: If we define information broadly as any quality of a building block that impacts the emergent behavior of a system of those particles (in this case, force direction and shape), then we can argue this work is looking at a few aspects of information in active systems.

\subsection{Mention work on NSF grant proposal?}
Using DNA-mediated assembly to store information in nanoparticle arrays. 

\section{Description of proposed research (7-8 pg, 2-3 pg per aim)}
% Including method or approach and expected difficulties
% This must constitute about 50\% of the text of the written proposal: 7-8 pages
% Clear statement of the work to be undertaken and must include:
% Objectives for the period of the proposed work and expected significance
% Relation to the present state of knowledge in the field and to work in progress at Michigan/elsewhere
% Expected research program sequence
% Decision points expected during the course of the research
% Methods of data reduction, evaluation, interpretation and presentation

High-level: ``Information'' are those factors that impact the yield and kinetics of self-assembly. To engineer them, we must continue to understand them.

Need to figure out what angle I want to approach this from:
\begin{itemize}
\item Pathway engineering?
\item``Smart''particle building blocks?
\item Algorithmically-designed interactions/patterned structures?
\item ... something else?
\end{itemize}

For each aim:
\begin{itemize}
\item Goal and significance
\item Hypothesis
\item Approach, methods, analysis to be used (including relevant citations)
\end{itemize}

\section{Time table}
% Basically a Gantt chart
\textit{Will put together a Gantt chart with associated milestones through 2020.}

\section{Conclusion}
Long, long-term goal: Instead of simply observing emergent behavior as an outcome of collective motion of individuals, we could instead engineer such behavior as a quantifiable outcome of the interaction of an information-rich network of agents.

\end{document}