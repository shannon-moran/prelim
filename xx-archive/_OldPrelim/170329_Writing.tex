\documentclass[11pt, oneside]{article}   	% use "amsart" instead of "article" for AMSLaTeX format
\usepackage{geometry}                		% See geometry.pdf to learn the layout options. There are lots.
\geometry{letterpaper}                   		% ... or a4paper or a5paper or ... 
%\geometry{landscape}                		% Activate for rotated page geometry
\usepackage[parfill]{parskip}    		% Activate to begin paragraphs with an empty line rather than an indent
\usepackage{graphicx}				% Use pdf, png, jpg, or eps§ with pdflatex; use eps in DVI mode
								% TeX will automatically convert eps --> pdf in pdflatex		
\usepackage{amssymb}

% Set text to 6 lines of 11pt font per 1 inch. Normal baseline skip is 13.6
% 1 inch = 72.27pt
% First line is 11pts
% Remainder spread as (61.27/5)/13.6 = 0.901
\linespread{0.901}

%SetFonts
%SetFonts

\begin{document}
Daily writing habit.

Working title: Working towards pluripotent materials

\section{Introduction and motivation - Non-equilibrium self-assembly}

\begin{itemize}
\item testting
\end{itemize}

Reconfigurable systems may be the key for unlocking adaptive material applications

Equilibrium self-assembly is useful for fabricating ordered structures on the nanoscale

However, equilibrium self-assembly processes aren't necessarily reliable 

Biology is really good at self-assembly, but it does so in a non-equilibrium manner in which systems are driven out of equilibrium by a constant input of energy (ATP)

However, the fundamental physics of such driven non-equilibrium self-assembly remains poorly understood

However, we have a problem-- we don't really understand non-equilibrium self-assembly, because we don't really understand non-equilibrium thermodynamics

KEY: 
Reconfigurability-induced switching of a material's structure could be explored as a way of storing information (``memory'') in a material

\section{Introduction and motivation - Information}
% Statement of the problem, purpose and significance of the research

Entropy is primarily an equilibrium construct. Information theory, however, is considerably broader. 

Open question: can these three distinct meanings of entropy-- information, number of accessible states, and heat-- be brought together in a meaningful way that would allow us to predictively and reliably encode information in materials systems?

1) What is entropy: Thermodynamics / heat - Entropy production out of equilibrium
- Energy is dissipated as entropy
- ``Entropy product is believed central to the behavior of non-equilibrium and dissipative systems, where structures are stabilized by the consumption of energy and the production of entropy
- General principles and predictive theories are not in hand because of our inability to mathematically formulate non-equilibrium variational principles and because we lack appropriate geometric and topological measures of entropy and information-- what would these even look like?

2) What is entropy: Statistical mechanics / number of accessible states - Equilibrium thermodynamics
- Boltzmann entropy; probability of being a given microstate is a function of the prevalence of that microstate

3) What is entropy: Information
- Story of Claude Shannon developing his information theory
- The fact that he named it entropy wasn't necessarily based on any explicit relationship to the thermodynamic or statistical mechanical form of entropy

4) In the granular materials community, ``memory'' is embedding some replicable response pattern in 

5) Engineering application: Designing functional, reconfigurable materials will require some method of storing information in a material-- we might call this ``memory''

6) Scientific application: Understanding how to do this is a fundamental problem
- Understanding the linkage between these three is critical to a deeper understanding of biochemical processes, out of equilibrium dynamics of macroscopic systems, and the ``dynamics'' part of thermodynamics

Key takeaway from information proposal: A coherent framework of thermodynamic and non-equilibrium processes seen through information theoretic eyes could lead to new theories for encoding information in matter-- which would allow for the design of novel materials and novel material behavioral control.

\section{Background - Literature review}
% Literature survey and description of research already performed by the applicant


\section{Background - Research already completed}
\subsection{Role of shape in collective motion}
Basically have two weeks to get something to go here...
\subsection{Maybe some shape contribution to the collision theory?}

\section{Description of proposed research (7-8 pages, 2-3 p per goal)}
% Including method or approach and expected difficulties
% This must constitute about 50\% of the text of the written proposal: 7-8 pages
% Clear statement of the work to be undertaken and must include:
% Objectives for the period of the proposed work and expected significance
% Relation to the present state of knowledge in the field and to work in progress at Michigan/elsewhere
% Expected research program sequence
% Decision points expected during the course of the research
% Methods of data reduction, evaluation, interpretation and presentation
\subsection{Aim 1}
For each aim:
\begin{itemize}
\item Goal and significance
\item Hypothesis
\item Approach, methods, analysis to be used (including relevant citations)
\end{itemize}

We must both define and ascertain the information content of a self-assembly pathway
If self-assembly proceeds deterministically along one route only, the entropy is clearly zero
However, optimal self-assembly processes are likely to have a high, maybe maximal ``path entropy''

To realize such information-rich structures in a wide range of chemical or colloidal structures, we must have a deeper, more rigorous understanding of the factors that optimize the yield and the kinetics of self-assembly

We could both solve the entropy problem and enable a completely new field based on the encoding of information in physical structures

\section{Time table}
% Basically a Gantt chart
\textit{Will put together a Gantt chart for each project with associated milestones. 	}

\section{Conclusion}
Instead of simply observing emergent behavior as an outcome of collective motion of individuals, we could instead engineer such behavior as a quantifiable outcome of the interaction of an information-rich network of agents.

\end{document}